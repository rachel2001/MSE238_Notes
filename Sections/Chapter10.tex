\documentclass[../main.tex]{subfiles}
\begin{document}

\section{Statistical Quality Control}

\textbf{Common causes/ chance causes:} difficult or impossible to specify.\\
If common causes are only causes of variation, the process is in  a state of statistical control/ in control.\\\\
\textbf{Special causes/ assignable causes: }unacceptable variation, caused by special factor.\\
If there are presence of one or more special causes, it is out of statistical control.\\\\
\textbf{Rational subgroup:} A number of samples taken over a period of time.\\\\
All the variability within the units in a rational subgroup should be due to common causes, non should be special causes.\\\\
\textbf{Sampling method:}\\
    1. Sample at regular time intervals, with all the items in each sample manufactured near the time the sampling is done.\\
    2. Sample at regular time intervals, with the items in each sample drawn from all the units produced since the last sample was taken.\\\\
Variable data: 3-8 unit, $>$20 samples\\
Binary and count data: larger

\subsection{Control Charts}

$\bar X$ approximates the process mean $\mu$ during the time its sample was taken.\\
R and s can be used to approximate the process standard deviation$\sigma$\\
If the process is in control, $\bar X$, R,s will almost always be within \textbf{control limits}.\\

\subsection{X–bar charts}

\begin{align*}
    3\sigma UpperLimit = \bar{\bar{X}}+A_2\bar{R} \\
    CenterLine=\bar{\bar{X}}\\
    3\sigma LowerLimit = \bar{\bar{X}} - A_2 \bar R
\end{align*}

\subsubsection*{X–bar and R charts}
\textbf{Steps:}\\
1. Choose rational subgroups.\\
2. Compute the R chart.\\
3. Determine the special causes for any out-of-control points.\\
4. Recompute the R chart, omitting samples that resulted in out-of-control points.\\
5. Once the R chart indicates a state of control, compute the $\bar X$ chart.\\
6. If the $\bar X$ chart indicates that the process is not in control, identify and correct any special causes.\\
7. Continue to monitor $\bar X$ and R.\\\\
R chart can be used to access variation in the \textbf{sample range};\\
R chart has a center line at $\bar R$ and upper and lower lines indicate the $3\sigma$ upper and lower control limits.
\begin{align*}
    3\sigma UpperLimit = D_4\bar R\\
    CenterLine=\bar R\\
    3\sigma LowerLimit = D_3\bar R
\end{align*}

\subsubsection*{X–bar and S charts}

S chart can be used to assess variation in the \textbf{sample standard deviation}.\\
S chart is an alternative to the R chart
\begin{align*}
    3\sigma UpperLimit = B_4\bar S\\
    CenterLine=\bar S\\
    3\sigma LowerLimit = B_3\bar S
\end{align*}
Attribute control chart: used for binary or count variables\\

\subsection{p charts: for binary variables}
The number of items in each sample is n.\\
\begin{align*}
    3\sigma UpperLimit = \bar p + 3\sqrt{\frac{\bar p (1-\bar p)}{n}}\\
    CenterLine=\bar{p}\\
    3\sigma LowerLimit = \bar p - 3\sqrt{\frac{\bar p (1-\bar p)}{n}}\\
    \bar p = \sum_{i=1}^{k}\frac{\hat{p_i}}{k}
\end{align*}
The control limits valid when $n\bar p >10$

\subsection{c charts: for count variables}
C-charts show how the process, measured by the number of nonconformities per item or group of items, changes over time.
The c chart is used when the quality measurement is a count of the number of defects, or flaws, in a given unit.\\
Use of the c chart requires the number of defects follow a Poisson distribution.\\
Assume k units are sampled, let $c_i$ denote the total number of defects in the $i^{th}$ unit.
$\lambda$ denote the mean total number of flaws per unit.
Then$c_i \thicksim Poisson(\lambda)$\\
\begin{align*}
    3\sigma UpperLimit = \bar c + 3\sqrt{\bar{c}}\\
    CenterLine=\bar{c}\\
    3\sigma LowerLimit = \bar c - 3\sqrt{\bar{c}}\\
    \bar{c} = \sum_{i=1}^{k} \frac{c_i}{k}
\end{align*}
Valid when $\bar c > 10$.

\subsection{Process Capability}
\textbf{Definition: }The ability of a process to produce output that meets a given specification\\
A process is in control if there are no special causes operating.\\
The values of the quality characteristic vary without any trend or pattern, since common causes do not change over time.\\
However, a process is possible to be in control and yet to be producing output that does not meet a given specification.\\
If a quality characteristic from a process in a stats of control is normally distributed:
\begin{align*}
    \hat{\mu} = \bar{\bar X}
\end{align*}
To be fit for use, a quality characteristic must fall between a lower specification limit and an upper specification limit.\\
They are not the control limits found on control charts.

\subsection{6-sigma Quality}
\textbf{Definition: }A process is said to have six sigma quality if the process capability index Cp has a value of 2.0 or greater.Equivalently, a process has six-sigma quality if the difference USL-LSL is at least 12 $\sigma$.\\
It can withstand moderate shifts in process mean without significant deterioration in capability.
\end{document}