\documentclass[../main.tex]{subfiles}
\begin{document}

\section{Point and Interval Estimation for a Single Sample}
The sample mean and sample proportion are examples of point estimates.\\
The purpose of a confidence interval is to provide a margin of error for the point estimate, to indicate how far off the true value it is likely to be.\\
In general, a quantity calculated from data is called a statistic, and a statistic that is used to estimate an unknown constant, or parameter, is called a point estimator or point estimate.\\
Statistics are often used to estimate parameters. \\
A statistic that is used to estimate a parameter is called a \textbf{point estimate}.\\
Random quantity $\hat{p} = X/n$, if $\xsim B(n,p)$
\\
The \textbf{bias} of an estimator is the difference between the mean of the estimator and the true value being estimated.\\
It is important to note that the bias only describes the closeness of the mean of the estimator.\\
Let $\theta$ be a parameter, $\hat \theta$ be an estimator of $\theta$.
\begin{equation*}
    \mbox{Bias} = \mu_{\hat{\theta}}-\theta
\end{equation*}
An estimator with bias of 0 is unbiased.\\\\
\textbf{Variance} measures spread. A small variance is better.\\\\
\textbf{Mean Squared Error} = variance + square of the bias.
\begin{equation*}
    MSE=(\mu_{\hat{\theta}}-\theta)^2+\sigma _{\hat{\theta}}^2 = \mbox{Mean of the square of the errors}
\end{equation*}
Many confidence interval = point estimate $\pm$ margin of error, where margin of error = (critical value)(standard error) and the critical value is the multiplier that gives the confidence level.\\
\subsection{Confidence intervals for one population mean}
Let \xotn be a large ($n>30$) random sample from a population with mean $\mu$ and standard deviation $\sigma$, so that $\xbar$ is approximately normal. Then a level $100(1-\alpha)\%$ confidence interval for $\mu$ is: 
\begin{equation*}
     \xbar \pm z_{\frac{\alpha}{2} , \sigma_{\xbar}}
\end{equation*}
When the value of $\sigma$ is unknown, $\sigmax$ can be replaced with sample standard deviation s.\\
\begin{center}
\begin{tabular}{ |c|c| } 
 \hline
 z & CI \\ 
 \hline
 1 & 68\% \\ 
 1.645 & 90\% \\ 
 1.96 & 95 \% \\
 2.58 & 99\% \\
 3 & 99.7\% \\
 \hline
\end{tabular}
\end{center}
e.g. $z_{0.5\alpha}=1.96$ and the estimated standard deviation is 5, then $0.5 = 1.96 \frac{5}{\sqrt{n}}$, n = 385.\\
Note: always round n up!!\\
\\
The sample size n needed to construct a level $100(1-\alpha)\%$ confidence interval of width $\pm w$ is:
\[
n=\frac{Z_{\alpha / 2}^2\sigma^2}{w^2}
\]


\subsection{Confidence intervals for one population proportion}
Let X be the number of success in n independent Bernoulli trails with success probability p, so that $Z\sim \mbox{Bin}(n,p)$, then a $100(1-\alpha)\%$ confidence interval for p is:
\[\tilde{p}\pm z_{{\alpha}/{2}}\sqrt{\frac{\tilde{p}(1-\tilde{p})}{\tilde{n}}}
\]
If lower limit $<$0, replace with 0; if upper limit $>$1, replace with 1.\\
\\
Determine sample size:
\[n=\frac{z_{\alphat}^2 \tilde{p}(1-\tilde{p})}{w^2}-4 \mbox{ if an estimate of } \tilde{p}\mbox{ is available}\]
\[n=\frac{z_{\alphat}^2 }{4w^2}-4 \mbox{ if no estimate of } \tilde{p}\mbox{ is available}\]
Traditional Method:\\
Let $\phat$ be the proportion of success in a large number of independent Bernoulli trails with success probability p.
The sample must contains at least 10 successes and 10 failures.\\
\[\phat\pm z_{{\alpha}/{2}}\sqrt{\frac{\phat(1-\phat)}{n}}
\]

\subsection{Student's t distribution}
t Distribution could be used when the sample size is small, and the population is approximately normal.\\
Let \xotn be a small random sample ($n<30$) from a normal population with mean $\mu$.
\[t=\frac{\xbar - \mu}{s/{\sqrt{n}}}\]
has a \textbf{student's t distribution} with \textbf{n-1 degrees of freedom} (denoted by $t_{n-1}$)\\
\\
The probability density of t distribution is different for different degrees of freedom.The t curves are more spread out than standard normal distribution.
When sample contains outlier, a reasonable way to proceed is to construct a boxplot or dotplot of the sample. If these plots do not reveal a strong asymmetry or any outliers, then in most cases the Student's t distribution will be reliable.\\
\\
Let \xotn be a small random sample from a normal population with mean $\mu$. \\
Confidence interval for $\mu$ is:
\[ \xbar \pm t_{n-1,\alphat}\frac{s}{\sqrt{n}} \]

\subsection{Prediction intervals and tolerance intervals}
A \textbf{prediction interval} is an interval that is likely to contain the value of an item sampled from a population at a future time.\\
If the shape of the population differs much from the normal curve, the prediction interval may be misleading.\\
The population needs to be normal.\\
\\
A \textbf{tolerance interval} is an interval that is likely to contain a specified proportion of the population.\\
Let \xotn be a sample from a normal population. A tolerance interval for containing $100(1-\gamma)\%$ of the population with confidence $100(1-\gamma)\%$ us:
\[
\xbar \pm k_{n,\alpha,\gamma}s
\]
\end{document}